%!TEX TS-program = xelatex
%!TEX encoding = UTF-8 Unicode
% Awesome CV LaTeX Template for Cover Letter
%
% This template has been downloaded from:
% https://github.com/posquit0/Awesome-CV
%
% Authors:
% George Abitbol <george@abitbol.org>
% Lars Richter <mail@ayeks.de>
% 
% Francisé et modifier par ordinatous
% Ludovic Marchal <contact@ordinatous.com>
%
% Template license:
% LICENCE CC BY-SA 4.0 (https://creativecommons.org/licenses/by-sa/4.0/)
%
%-------------------------------------------------------------------------------
% CONFIGURATIONS du document
%-------------------------------------------------------------------------------
% FORMAT du papier
% Format A4 par défaut, utiliser 'letterpaper' pour le format US letter:
\documentclass[11pt, a4paper]{awesome-cv}
% LANGUE
% Francisation du document , format de date, syntaxe et correcteur orthographique:
\usepackage[french]{babel}
% MARGES
% Configurer les marges ici:
\geometry{left=1.4cm, top=.8cm, right=1.4cm, bottom=1.8cm, footskip=.5cm}
% FONTS
% Indiquer l'emplacement des fonts ici:
\fontdir[..fonts/]
% COULEUR
% Couleur pour les mise en avant:
% Awesome Couleurs: awesome-emerald, awesome-skyblue, awesome-red, awesome-pink, awesome-orange
%                 awesome-nephritis, awesome-concrete, awesome-darknight
\colorlet{awesome}{awesome-red}
% Décommenter pour choisir d'autres couleurs:
% \definecolor{awesome}{HTML}{CA63A8}

% Coleurs pour le texte:
% Décommenter pour en choisir une:
% \definecolor{darktext}{HTML}{414141}
% \definecolor{text}{HTML}{333333}
% \definecolor{graytext}{HTML}{5D5D5D}
% \definecolor{lighttext}{HTML}{999999}
% ACTIVER Highlights
% Basculer la valeur à false pour suprimmer les mise en avant:
\setbool{acvSectionColorHighlight}{true}

% If you would like to change the social information separator from a pipe (|) to something else
\renewcommand{\acvHeaderSocialSep}{\quad\textbar\quad}


%-------------------------------------------------------------------------------
%	INFORMATIONS PERSONNELLE
%	Commenter les entrées suivantes si elles ne sont pas nécessaires:
%-------------------------------------------------------------------------------
% Available options: circle|rectangle,edge/noedge,left/right
%\photo{../examples/profil_02}
\name{Cécile}{Sommen}
\position{PhD - Data-Scientist / Biostatisticienne}
\address{1B rue du Temple, 94100 Saint-Maur-des-Fossés}

\mobile{0685845559}
\email{cecile.sommen@proton.me}
% RESEAUX SOCIAUX
\github{cecilesommen}
\linkedin{cécile-sommen-101a70167}
% \gitlab{gitlab-id}
% \stackoverflow{SO-id}{SO-name}
% \twitter{@twit}
% \skype{skype-id}
% \reddit{reddit-id}
% \medium{madium-id}
% \googlescholar{googlescholar-id}{name-to-display}
%% \firstname and \lastname will be used
% \googlescholar{googlescholar-id}{}
% \extrainfo{extra informations}
% CITATION
%\quote{« Attention ! ce flim n'est pas un flim sur le « cyclimse ». Merci de votre compréhension »}


%-------------------------------------------------------------------------------
%	INFO du courrier
%	All of the below lines must be filled out
%-------------------------------------------------------------------------------
% INTERLOCUTEUR
\recipient
  {Santé publique France}
  {12 rue du Val d'Osne, 94415 Saint Maurice Cedex}
% The date on the letter, default is the date of compilation
\letterdate{Le \today}
% TITRE
\lettertitle{Candidature au poste de Data Engineer / Scientist (f/h) (réf. DATA-CDI-2026-03)}
% INTRO

\letteropening{}
% OUTRO
\letterclosing{Je vous prie d’agréer, Madame, Monsieur, mes salutations distinguées.}
% Any enclosures with the letter
%\letterenclosure[Veillez trouver en pièce jointe]{Curriculum Vitae}


%-------------------------------------------------------------------------------
\begin{document}

% Print the header with above personal informations
% Give optional argument to change alignment(C: center, L: left, R: right)
\makecvheader[L]

% Print the footer with 3 arguments(<left>, <center>, <right>)
% Leave any of these blank if they are not needed
\makecvfooter
  {Le \today}
  {Cécile Sommen~~~·~~~Candidature au poste de Data Engineer / Scientist (f/h) (réf. DATA-CDI-2026-03}
  {}

% Print the title with above letter informations
\makelettertitle

%-------------------------------------------------------------------------------
%	CONTENU du courrier
%-------------------------------------------------------------------------------
\begin{cvletter}


\vspace{0.5ex}

Madame, Monsieur,

\vspace{0.5ex}

<<<<<<< HEAD
Forte d’une double expertise en biostatistiques (PhD) et en data science (Mines Paris-PSL), et d’une expérience de 20 ans dont 12 dans des équipes DATA de Santé publique France,  je considère que ce poste s’inscrit dans la continuité logique de mon parcours professionnel. Il prolonge les missions que j’exerce depuis plusieurs années en matière de structuration, d’automatisation et de développement méthodologique des activités data.
=======
%Forte d’une double expertise en biostatistiques (PhD) et en data science (Mines Paris-PSL), et d’une expérience de 20 ans dont 12 dans des équipes DATA de Santé publique France, ce poste s’inscrit dans la continuité logique de mon parcours professionnel. Il prolonge les missions que j’exerce depuis plusieurs années en matière de structuration, d’automatisation et de développement méthodologique des activités data.
%
%\vspace{0.5ex}
%
%Mon expertise en biostatistiques et en épidémiologie des maladies infectieuses m’a conduite à intervenir sur des travaux nationaux et européens relatifs à l’estimation d’indicateurs (VIH, grippe, enquêtes de surveillance). J’ai notamment représenté la France lors de groupes de travail méthodologiques de l'ECDC entre institutions européennes, participant aux discussions scientifiques et à l’harmonisation des approches d’estimation de l'incidence du VIH.
%\vspace{0.5ex}
%
%Au sein de la DATA, j’ai développé une expertise couvrant l’ensemble du cycle de vie des données : coordination transverse, structuration de pipelines, automatisation et garantie de la qualité et de la traçabilité des traitements. J’ai conçu et industrialisé des chaînes complètes de production d’indicateurs nationaux, de l’ingestion des données à leur restitution automatisée (R, GitLab, Airflow). 
% Lors de la crise Covid-19, j’ai repensé et automatisé la production des indicateurs hospitaliers (SIVIC), permettant une diffusion fiable, documentée et sans intervention manuelle, dans des délais contraints et sous forte pression. Ma capacité à organiser, prioriser et sécuriser les processus vise en permanence à garantir des livrables robustes, reproductibles et rendus dans les temps impartis. Anticipant les évolutions structurelles que la crise rendait nécessaires, j’ai fait le choix d’approfondir et de formaliser mes compétences par une formation solide en Data Science / Data Ingénierie. 
%% C'est à ce moment que j'ai pris conscience que Santé publique France allait évoluer et tirer les enseignements de cette crise sanitaire pour gagner en agilité et je voulais être en mesure d'accompagner cette évolution. C'est cette intuition qui m'a poussée à  compléter mon expérience "autodidacte" par une formation solide en Data Science / Data Ingénieurie.
%
%\vspace{0.5ex}
%
%Habituée aux environnements exigeant réactivité et coordination multi-acteurs, je sais piloter des projets data en lien étroit avec les équipes métiers et techniques, arbitrer les priorités et structurer le travail collectif. Mon expérience m’a conduite à animer des échanges méthodologiques transversaux, à formaliser des procédures et à promouvoir l’usage de bonnes pratiques (versionning, documentation, automatisation), dans une logique de qualité et d’amélioration continue.
%
%\vspace{0.5ex}
%
%Depuis 2024, je travaille avec l’Institut Pasteur du Cambodge, où j’accompagne les équipes locales dans leurs analyses et le renforcement de leurs compétences. Je collabore également avec les directions métiers composées de profils internationaux et j'accompagne les doctorants et post-doctorants dans leurs analyses. J’y mobilise à la fois expertise technique et qualités pédagogiques : explicitation des choix méthodologiques, structuration des codes, transfert de bonnes pratiques sous R avec une adaptation aux profils et aux contextes culturels. J’interviens dans un contexte de forte exigence des bailleurs internationaux, impliquant production rapide de résultats et arbitrages méthodologiques sous contrainte. Cette expérience internationale multicuturelle a renforcé mon agilité, ma capacité d’adaptation et mon sens du collectif.
%
%\vspace{0.5ex}
%
%Curieuse et force de proposition, j’ai su évoluer vers des approches d’ingénierie et de science des données (Python, SQL, Spark, orchestration, data visualisation), tout en conservant une rigueur statistique forte et une compréhension fine des enjeux de santé publique. Cette double culture me permet d’articuler vision stratégique, exigences scientifiques et contraintes opérationnelles.
%
%\vspace{0.5ex}
%
%Investir ce poste représente pour moi l’opportunité de contribuer activement à la structuration, à la montée en qualité et en agilité des activités Data Science et Data Ingénierie de la direction DATA. Je serais ravie de pouvoir échanger avec vous afin de détailler la manière dont je pourrais accompagner cette dynamique.
\vspace{0.5ex}


Forte d’une double expertise en biostatistiques (PhD) et en data science (Mines Paris-PSL), et de 20 ans d’expérience dont 12 au sein des équipes DATA de Santé publique France, ce poste s’inscrit dans la continuité de mon parcours. Il prolonge les missions que j’exerce en structuration, automatisation et développement méthodologique des activités data.
>>>>>>> 5b5edb8 (modifs CV et lettre de motiv)

\vspace{0.5ex}

Mon expertise biostatistique sur le VIH et la grippe est reconnue aux niveaux national et international. Elle m’a conduite à représenter SpF au sein de groupes de travail nationaux (ANRS, Institut Pasteur, Réseau Sentinelles) et internationaux tels que l’ECDC et l’OMS, contribuant aux échanges scientifiques et à l’harmonisation des méthodes d’estimation.

\vspace{0.5ex}

Au sein de la DATA, j’ai développé une expertise couvrant l’ensemble du cycle de vie des données : coordination transverse, structuration de pipelines, automatisation et garantie de la qualité des traitements. J’ai industrialisé des chaînes complètes de production d’indicateurs nationaux, de l’ingestion à la restitution automatisée (R, GitLab, Airflow). Lors de la crise Covid-19, j’ai repensé la production des indicateurs hospitaliers (SIVIC), assurant une diffusion fiable et sans intervention manuelle dans des délais contraints. Anticipant les évolutions structurelles induites par cette crise, j’ai choisi de consolider cette expertise par une formation en Data Science / Data Ingénierie. Dans ces contextes exigeants et multi-acteurs, je structure priorités, méthodes et pratiques de qualité afin de sécuriser les projets data.

\vspace{0.5ex}

<<<<<<< HEAD
Depuis 2024, je développe une activité de data scientist à l’Institut Pasteur du Cambodge, où j’accompagne les équipes locales dans leurs analyses et le renforcement de leurs compétences. Je collabore également avec les directions métiers composées de profils internationaux et j'accompagne les doctorants et post-doctorants dans leurs analyses. J’y mobilise à la fois expertise technique et qualités pédagogiques : explicitation des choix méthodologiques, structuration des codes, transfert de bonnes pratiques sous R avec une adaptation aux profils et aux contextes culturels. Cette expérience internationale multicuturelle a renforcé mon agilité, ma capacité d’adaptation et mon sens du collectif.
=======
Depuis 2024, je travaille avec l’Institut Pasteur du Cambodge, où j’accompagne équipes locales, directions internationales et jeunes chercheurs dans leurs analyses. J’interviens dans un contexte de forte exigence des bailleurs, impliquant production rapide de résultats et arbitrages méthodologiques sous contrainte. Cette expérience multiculturelle a renforcé mon agilité et mon sens du collectif.
>>>>>>> 5b5edb8 (modifs CV et lettre de motiv)

\vspace{0.5ex}

Curieuse et force de proposition, j’ai élargi mon champ vers l’ingénierie et la science des données (Python, SQL, Spark, orchestration, visualisation), tout en conservant une rigueur statistique forte. Cette double culture me permet d’articuler vision stratégique, exigences scientifiques et contraintes opérationnelles.

\vspace{0.5ex}

Investir ce poste représente pour moi l’opportunité de contribuer à la structuration et à la montée en qualité des activités Data Science et Data Ingénierie de la direction DATA. Je serais ravie d’échanger avec vous à ce sujet.

\vspace{0.5ex}



\end{cvletter}


%-------------------------------------------------------------------------------
% Print the signature and enclosures with above letter informations
\makeletterclosing

\end{document}